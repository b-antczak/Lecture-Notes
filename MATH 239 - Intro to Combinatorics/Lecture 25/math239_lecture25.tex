\documentclass{report}
\usepackage[margin=1in, paperwidth=8.5in, paperheight=11in]{geometry}
%Math packages%
\usepackage{amsmath}
\usepackage{amssymb}
%\usepackage{MnSymbol}
\usepackage{amsthm}
%Spacing%
\usepackage{setspace}
\onehalfspacing
%Lecture number%
\newcommand{\lectureNum}{25}
%Variables - Date and Course%
\newcommand{\curDate}{March 8, 2017}
\newcommand{\course}{MATH 239}
\newcommand{\instructor}{Luke Postle}
%Defining the example tag%
%\theoremstyle{definition}%
\newtheorem{ex}{Example}[section]
%Setting counter given the lecture number%
\setcounter{chapter}{\lectureNum{}}
%Package for drawing graphs%
\usepackage{tikz}
\usepackage{verbatim}
\usetikzlibrary{arrows}

\begin{document}
%Note title%
\begin{center}
\begin{Large}
\textsc{\course{} | Lecture \lectureNum{}}
\end{Large}
\end{center} 
\noindent \textit{Bartosz Antczak} \hfill
\textit{Instructor: \instructor{}} \hfill
\textit{\curDate{}}
\rule{\textwidth}{0.4pt}
%Actual Notes%
\section{Generating Series}
\subsubsection{Definition}
The \textbf{generating series} of a set $A$ (e.g., binary string, compositions, etc.) equipped with a weight function\footnote{a weight function must be non-negative and an integer} $w$ (e.g., length, number of parts, etc.) is the formal power series whose coefficient of $x^n$ is the number of elements of $A$ of weight $n$.
This is denoted as
$$\Phi_A (x)$$
Now, $\Phi_A (x) = \displaystyle\sum_{n \geq 0} a_nx^n$, where $a_n$ is the number of elements of $A$ of weight $n$. Equivalently, $$\Phi_A (x) = \sum_{a \in A}x^{w(a)}$$
\begin{ex}
\end{ex}
Let $B$ be the set of binary strings, and $w$ be the length. The generating series is
$$\Phi_B (x) = 1 + 2x + 4x^2 + 8x^3 + \cdots$$
To write out the coefficients, we start with the number of possible strings there are of length 0 (which is 1), then the number of strings of length 1 (which is 2), and then we have 4, and then 8, and so on.
\begin{ex}
\end{ex}
Let $C$ be a set of compositions, and let the weight sum up to $n$ iff 
We have
$$\Phi_C(x) = 1 + 1x + 2x^2 + 4x^3 + 8x^4 + \cdots$$
\begin{ex}
\end{ex}
Let $S_m$ be the subsets of $[m]$ (for some fixed $m$). Let the weight be the size of the subset. We have
$$\Phi_{S_{m}} (x) =  1 + mx + {m \choose 2}x^2 + {m \choose 3}x^3 + \cdots + x^m \qquad \text{(i.e., up until we reach} {m \choose m})$$
But there is actually a more compact way to write it:
$$\Phi_{S_{m}}(x) = \sum_{n=0}^m {m \choose n}x^n = (1+x)^m \qquad\text{(By binomial theorem)}$$
In this class, we let
\begin{align*}
\mathbb{N} &= \{1, 2, 3, \cdots\} \\
\mathbb{N}_0 &= \{0, 1, 2, 3, \cdots\} \\
\mathbb{N}_k &= \{k, k+1, k+2, \cdots\}
\end{align*}
\begin{ex}
\end{ex}
What are the generating series for $\mathbb{N}, \mathbb{N}_0, \mathbb{N}_k$ with $w(i) = i$?
\begin{align*}
\Phi_{\mathbb{N}} (x) &= x + x^2 + x^3 + \cdots \\
\Phi_{\mathbb{N}_0} (x) &= 1 + x + x^2 + x^3 + \cdots \\
\Phi_{\mathbb{N}_k} (x) &= x^k + x^{k+1} + x^{k+2} + \cdots
\end{align*}
What about $\mathbb{N}$ but with $w(i) = i^2$?
\begin{align*}
\Phi_{\mathbb{N}}(x) &= 0 + 1x + 0x^2 + 0x^3 + 1x^4 + \cdots  \\
&= x + x^4 + x^9 + x^{16} = \cdots
\end{align*}
\begin{ex}
\end{ex}
Consider the Cartesian product of $A = \{1,3,5\} \times \{2,4,7\}$. Define the weight function $w((a,b)) = a=b$. We can write the generating series using ``brute force":
$$\Phi_{A}(x) = x^{1+2} + x^{1+4} + x^{1+7} + x^{3+2} + x^{3+4} + x^{3+7} + x^{5+2} + x^{5 + 4} + x^{5+7}$$
Observe that we can factor this line:
$$(x^1 + x^3 + x^5)(x^2 + x^4 + x^7)$$
We'll cover this more on Friday.
\section{Power Series}
\subsubsection{Definition}
If $(a_0, a_1, \cdots)$ is a sequence of rational numbers, then
$$A(x) = \sum_{n \geq 0} a_nx^n$$
is a \textbf{formal power series}. We let $[x^n]A(x)$ denote the coefficients of $x^n$, in other words, $a_n$. \newpage
\begin{ex}
\end{ex}
Solve $[x^2](1+x)^5$. In other words, find the coefficients of $x^2$ in the power series $(1+x)^5$.\\Here, we want to solve
$$[x^2]\sum_{n=0}^5 {5 \choose n}x^n \qquad\text{(By theorem)}$$
when $n=2$. Thus, $[x^2](1+x)^5 = {5 \choose 2} = 10.$
\subsection{Operations on Formal Power Series}
Let $A(x) = \displaystyle\sum_{n \geq 0}a_nx^n$, $B(x) = \displaystyle\sum_{n \geq 0}b_nx^n$ be formal power series. The operations we perform on them are outlined:
\begin{itemize}
\item \textbf{Addition:} $A(x) + B(x) = \displaystyle\sum_{n \geq 0} (a_n + b_n)x^n$
\item \textbf{Subtraction:} $A(x) - B(x) = \displaystyle\sum_{n \geq 0} (a_n - b_n)x^n$
\item \textbf{Multiply by constant:} $kA(x) = \displaystyle\sum_{n \geq 0} (ka_n)x^n$
\item \textbf{Multiplication of Power Series:} $A(x) \cdot B(x) = \displaystyle\sum_{n \geq 0} \left(\sum_{i=0}^n a_ib_{n-i}\right)x^n$\\From this we see that $$[x^n]A(x)\cdot B(x) = \sum_{i=0}^n a_ib_{n-i}$$
\begin{ex}
\end{ex}
Consider $A(x) = B(x) = 1 + x + x^2 + \cdots$:
$$A(x) \cdot B(x) = 1 = 2x + 3x^2 + \cdots + (n+1)x^n$$ We also see that
$$[x^n]A(x)\cdot B(x) = \sum_{i=0}^n a_ib_{n-i} = \sum_{i=0}^n 1 \cdot 1 = n+1$$
\item \textbf{Division:} recall the definitions of the \textbf{inverse}.
\begin{ex}
\end{ex}
Let $B(x) = 1-x$. The inverse, or $\frac{1}{B(x)}$, is equal to $ 1 + x + x^2 + \cdots$ 
\end{itemize}
\subsubsection{Note about Formal Power Series}
We don't care what the value of $x$ is. \textit{We are only concerned about the coefficients.}
%END%
\end{document}