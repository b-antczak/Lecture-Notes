\documentclass{report}
\usepackage[margin=1in, paperwidth=8.5in, paperheight=11in]{geometry}
%Math packages%
\usepackage{amsmath}
\usepackage{amssymb}
%\usepackage{MnSymbol}
\usepackage{amsthm}
%Spacing%
\usepackage{setspace}
\onehalfspacing
%Lecture number%
\newcommand{\lectureNum}{22}
%Variables - Date and Course%
\newcommand{\curDate}{March 1, 2017}
\newcommand{\course}{MATH 239}
\newcommand{\instructor}{Luke Postle}
%Defining the example tag%
%\theoremstyle{definition}%
\newtheorem{ex}{Example}[section]
%Setting counter given the lecture number%
\setcounter{chapter}{\lectureNum{}}
%Package for drawing graphs%
\usepackage{tikz}
\usepackage{verbatim}
\usetikzlibrary{arrows}

\begin{document}
%Note title%
\begin{center}
\begin{Large}
\textsc{\course{} | Lecture \lectureNum{}}
\end{Large}
\end{center} 
\noindent \textit{Bartosz Antczak} \hfill
\textit{\curDate{}}
\rule{\textwidth}{0.4pt}
%Actual Notes%
\subsubsection{Note}
\textit{I was away for today's lecture, so these notes are my summaries from the course notes, from section 1.4}
\section{Bijections}
From last lecture, we mentioned a proposition:
\begin{center}
\textit{If there exists a bijection from A to B, then $|A| = |B|$}
\end{center}
\subsection{Explanation of our Proposition}
So this proposition will be handy when proving that two sets are equal. So how can we show a bijection? Well, we'll do so by providing a function and an inverse function that relates two sets together. If there exists such a function and inverse, then there exists a bijection between the sets and they're equal. We rewrite the previous statements as a theorem:
\begin{center}
\textit{If function $f : S \rightarrow T$ has an inverse, then f is a bijection}
\end{center}
\begin{ex}
For some $0 \leq k \leq n$, let S be the set of k-subsets of $\{1, \cdots, n\}$, and let T be the set of (n-k)-subsets of $\{1, \cdots, n\}$. Find a bijection between S and T.
\end{ex}
We can do the following mapping:
$$f: S \rightarrow T : f(A) = \{1, \cdots, n\} - A \quad \forall A \in S$$
Since $A \in S$, $|A| - S$, and we also see that $f(A) \ \{1, \cdots, n\} - A$ has $n-k$ cardinality, thus $f(A) \in T$.\\In order to ``reverse" the operation, we simply need to apply the same operation again, but this time we're applying the operation on the output. So the inverse is $f^{-1}: T \rightarrow S$ where for each $B \in T$, $f^{-1}(B) = \{1, \cdots, n\} - B$. We check that for each $A \in S$:
$$f^{-1}(f(A)) = f^{-1}(\{1, \cdots, n\} - A) = \{1, \cdots, n\} - (\{1, \cdots, n\}- A) = A$$
And for each $B \in T$:
$$f(f^{-1}(B)) = f(\{1, \cdots, n\} - B) = \{1, \cdots, n\} - (\{1, \cdots, n\}- B) = B$$
So $f^{-1}$ is indeed the inverse of $f$ which proves that $f$ is a bijection, ergo, both sets are of equal size.
%END%
\end{document}