\documentclass{report}
\usepackage[margin=1in, paperwidth=8.5in, paperheight=11in]{geometry}
%Math packages%
\usepackage{amsmath}
\usepackage{amssymb}
%\usepackage{MnSymbol}
\usepackage{amsthm}
%Spacing%
\usepackage{setspace}
\onehalfspacing
%Lecture number%
\newcommand{\lectureNum}{26}
%Variables - Date and Course%
\newcommand{\curDate}{March 10, 2017}
\newcommand{\course}{MATH 239}
\newcommand{\instructor}{Luke Postle}
%Defining the example tag%
%\theoremstyle{definition}%
\newtheorem{ex}{Example}[section]
%Setting counter given the lecture number%
\setcounter{chapter}{\lectureNum{}}
%Package for drawing graphs%
\usepackage{tikz}
\usepackage{verbatim}
\usetikzlibrary{arrows}

\begin{document}
%Note title%
\begin{center}
\begin{Large}
\textsc{\course{} | Lecture \lectureNum{}}
\end{Large}
\end{center} 
\noindent \textit{Bartosz Antczak} \hfill
\textit{Instructor: \instructor{}} \hfill
\textit{\curDate{}}
\rule{\textwidth}{0.4pt}
%Actual Notes%
\subsubsection{Review}
Negative binomial series (remember this for the exam):
$$\frac{1}{(1-x)^k} = \sum_{n \geq 0} {n+k - 1 \choose k-1}x^n$$
\section{The Inverse of Power Series}
We state a theorem:
\subsection{Theorem 1}
\begin{center}
\textit{$B(x)$ has an inverse $\iff$ $b_0 \neq 0$}
\end{center}
\subsubsection{Proof of Theorem 1}
\begin{itemize}
\item \textbf{($\implies$):}\\If $b_0 = 0$, then $B(x)$ has no inverse $D(x)$ such that $D(x) \cdot B(x) = 1 \quad \qed$
\item \textbf{($\impliedby$):}\\Suppose $b_0 \neq 0$. We want to find $D(x)$ such that $D(x) \cdot B(x) = 1$. With $D(x) \cdot B(x) = 1$ we want:
$$d_0b_0 + (d_0b_1 + d_1b_0)x + (d_0b_2 + d_1b_1 d_2b_0)x^2 + \cdots = 1 + 0x + 0x^2 + \cdots$$
But then the coefficients must be equal, giving rise to an infinite number of equations:
\begin{itemize}
\item Constant: $d_0b_0 = 1$
\item Linear: $d_0b_1 + d_1b_0 = 0$
\item Quadratic: $d_0b_2 + d_1b_1 + d_2b_0 = 0$
\end{itemize}
From these equations, the $d_i$ are uniquely determined as follows:
\begin{align*}
d_0 &= \frac{1}{b_0} \\
d_1 &= -\frac{b_1}{b_0^2} \\
d_0 &= \frac{-d_0b_2 - d_1b_1}{b_0}
\end{align*}
Or to generalize
$$d_0 = \frac{1}{b_0}$$
$$\forall \, n \geq 1 \quad d_n = \frac{-(d_{n-1}b_1 + d_{n-2}b_2 + \cdots + d_0b_{n-1})}{b_0} \qquad \qed$$
\end{itemize} 
\subsection{Corollary 1}
\begin{center}
\textit{If the inverse of a formal power series exists, the it's unique. Moreover, the inverse of the inverse is itself}
\end{center}
\begin{ex}
\end{ex}
Consider $\frac{1}{1-2x}$ and $\frac{1}{1-x^2}$? Do these expressions have an inverse? They \textbf{do}, since $b_0 \neq 0$.
\subsection{Definition | Composition}
The \textbf{composition} of $A(x)$ and $B(x)$ is $$A(B(x)) = a_0 + a_1B(x) + a_2B(x)^2 + \cdots$$
\subsubsection{Question}
When are compositions well-defined? Particularly, considering its sum up to infinity (for instance, $\displaystyle\sum_{i=1}^\infty \frac{1}{x^2}$ is well-defined, whereas $\displaystyle\sum_{i=0}^\infty (-1)^n$ is not).\\
\subsection{Theorem 2}
\begin{center}
\textit{$A(B(x))$ is well-defined $\iff$ either $A(x)$ is finite \underline{or} $b_0 = 0$}
\end{center}
\subsubsection{Proof}
\begin{itemize}
\item If $A(x)$ is finite, then $A(B(x))$ is the finite sum of a formal power series, and so it's well-defined
\item If $b_0 = 0$, then the smallest term in $B(x)^n$ has degree at least $n$ (since the constant is zero), but then
\begin{align*}
[x^n]A(B(x)) &= \sum_{m \geq 0}[x^n]a_mB(x)^m \\
&= \sum_{m = 0}^n [x^n]a_mB(x)^m
\end{align*}
because the $n$th coefficient of $B(x)^m$ is zero if $m > n$ (so the rest of the sum is zero, and thus it is a finite sum, ergo it's well-defined). $\qquad \qed$
\end{itemize}
\newpage
\begin{ex}
\end{ex}
Consider $A(x) = \frac{1}{1-x}$ and $B(x) = 2x$. $A(B(x))$ is well-defined because $b_0 = 0$:
\begin{align*}
\frac{1}{1-2x} = A(B(x)) &= \sum_{n \geq 0}a_nB(x)^n \\
&= \sum_{n \geq 0}1\cdot(2x)^n  & (\text{Since } A(x) = 1 + x + x^2 + \cdots) \\
&= \sum_{n \geq 0} 1 \cdot 2^n \cdot x^n \\
&= \sum_{n \geq 0} 2^n \cdot x^n
\end{align*}
We observe that this is the generating series for binary strings with the weight being the length of the string, so:
$$\Phi_B(x) = \sum_{n \geq 0} 2^nx^n = \frac{1}{1-2x}$$
\section{Sum and Product Lemmas}
\subsection{Sum Lemma}
\begin{center}
\textit{If $B = A_1 \sqcup A_2$. where they all have the same weight function $w$, then $\Phi_B(x) = \Phi_{A_1}(x) + \Phi_{A_2}(x)$}
\end{center}
\subsubsection{Proof of Sum Lemma}
Let $\Phi_B(x) = \displaystyle\sum_{n \geq 0}b_nx^n$, $\Phi_{A_1}(x) = \displaystyle\sum_{n \geq 0}a_{1,n}x^n$, $\Phi_{A_2}(x) = \displaystyle\sum_{n \geq 0}a_{2, n}x^n$. By definition, $b_n$ is equal to the number of elements of $B$ with weight $n$. Since $B$ is the disjoint union of $A_1$ and $A_2$, then $b_n$ is also equal to the sum of the number of elements in $A_1$ and $A_2$ of weight $n$, which is defined by $a_{1,n} + a_{2,n}$. Ergo
$$\Phi_B(x) = \Phi_{A_1}(x) + \Phi_{A_2}(x)$$
\subsection{Product Lemma}
\begin{center}
\textit{If $B = A_1 \times A_2$ (the Cartesian product) and $w((a_1, a_2)\in B) = w(a_1) + w(a_2)$, then $\Phi_B(x) = \Phi_{A_1}(x) \cdot \Phi_{A_2}(x)$}
\end{center}
%END%
\end{document}