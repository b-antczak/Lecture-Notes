\documentclass{report}
\usepackage[margin=1in, paperwidth=8.5in, paperheight=11in]{geometry}
%Math packages%
\usepackage{amsmath}
\usepackage{amsthm}
%Spacing%
\usepackage{setspace}
\onehalfspacing
%Lecture number%
\newcommand{\lectureNum}{3}
%Variables - Date and Course%
\newcommand{\curDate}{January 25, 2017}
\newcommand{\course}{MATH 239}
\newcommand{\instructor}{}
%Defining the example tag%
%\theoremstyle{definition}%
\newtheorem{ex}{Example}[section]
%Setting counter given the lecture number%
\setcounter{chapter}{\lectureNum{}}
%Package for drawing graphs%
\usepackage{tikz}
\usepackage{verbatim}
\usetikzlibrary{arrows}

\begin{document}
%Note title%
\begin{center}
\begin{Large}
\textsc{\course{} | Tutorial \lectureNum{}}
\end{Large}
\end{center} 
\noindent \textit{Bartosz Antczak} \hfill
\textit{\curDate{}}
\rule{\textwidth}{0.4pt}
% Actual Notes%
\subsubsection{Deriving a Formula}
Let $n_t$ denote the number of vertices of degree $t$ in a tree:
$$n_1 = 2 + \sum_{r = 3}^n (r-2)n_r$$
This formula was outlined in the lecture 8 notes, subsection 8.1.1-8.1.2. Note, this formula \textit{only} holds true for any graph that satisfies $\vert V(G) \vert = \vert E(G) \vert  + 2$ (which is usually a tree).\\
We'll cover some questions from the course notes, in problem set 5.1.
\section{Problem Set 5.1}
\subsubsection{3. What is the smallest number of vertices of degree 1 in a tree with 3 vertices of degree 4 and 2 vertices of degree 5?}
We know that $n_4 = 3$ and $n_5 = 2$.
\begin{align}
n_1 &= 2 + \sum_{r = 4}^5 n_r(r-2) \\
&= 2 + 3(4-2) + 2(5-2)\\
&= 14
\end{align}
Therefore, there must be at least 14 vertices of degree 1. A possible drawing of this tree is shown:
%TODO draw tree?%
\subsubsection{5. A cubic tree is a tree where all vertices have degree 3 or 1. Prove that a cubic tree with exactly k vertices of degree 1 has 2(k-1) vertices.}
Referring to our formula again
\begin{align}
n_1 &= 2 + \sum_{r = 3}^n n_r(r-2) \\
k &= 2 + n_3(3-2) \\
n_3 &= k-2
\end{align}
Observe that $\vert V(T) \vert = n_1 + n_3 = (k + k - 2) = 2(k-1)$, as desired.
\subsubsection{6. A forest is a graph with no cycles. Prove that a forest with p vertices and q edges has p-q components}
Let $F$ be a forest. Let $F = \{T_1, T_2, \cdots, T_k\}$ (where $T_i$ is a tree for all $i \in [1,k]$). We also know that for every tree, $\vert V(T_i) \vert = 1 + \vert E(T_i) \vert$. Observe
\begin{align}
\sum_{i=1}^k \vert V(T_i) \vert &= \sum_{i=1}^k (1 + \vert E(T_i) \vert) \\
\vert V(F) \vert &= k + \vert E(F) \vert\\
k &= \vert V(F) \vert-\vert E(F) \vert \\
&= p - q
\end{align}
%END%

\end{document}