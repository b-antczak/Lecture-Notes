\documentclass[10pt]{report}
\usepackage[margin=1in, paperwidth=8.5in, paperheight=11in]{geometry}
%Math packages%
\usepackage{amssymb}
\usepackage{amsmath}
\usepackage{amsthm}
%Spacing%
\usepackage{setspace}
\onehalfspacing
%Lecture number%
\newcommand{\lectureNum}{11}
%Variables - Date and Course%
\newcommand{\curDate}{March 29, 2017}
\newcommand{\course}{MATH 239}
\newcommand{\instructor}{}
%Defining the example tag%
%\theoremstyle{definition}%
\newtheorem{ex}{Example}[section]
%Setting counter given the lecture number%
\setcounter{chapter}{\lectureNum{}}
%Package for drawing graphs%
\usepackage{tikz}
\usepackage{verbatim}
\usetikzlibrary{arrows}
\def\part#1{\item[\bf #1)]}

\begin{document}
%Note title%
\begin{center}
\begin{Large}
\textsc{\course{} | Tutorial \lectureNum{}}
\end{Large}
\end{center} 
\noindent \textit{Bartosz Antczak} \hfill
\textit{\curDate{}}
\rule{\textwidth}{0.4pt}
% Actual Notes%
\section*{Problem Set 3.2}
\begin{enumerate}
\item[1.]
\begin{enumerate}
\item \textit{Consider $c_n = 5c_{n-1}-3c_{n-2}-9c_{n-3}$, $n \geq 3$ with initial conditions $c_0=1, c_1=1, c_2 = 29$. Find $c_n$ explicitly}
\subsubsection{Solution}
Rewrite it as $c_n - 5c_{n-1}+3c_{n-2}+9c_{n-3} = 0$. By theorem 3.2.1, we define $Q(x) = 1-5x+3x^2+9x^3$. We define our characteristic polynomial as $C(x) = x^3 - 5x^2 + 3x + 9 = (x+1)(x-3)^2$. Here, we have two roots $-1,\;3$ with multiplicities $1, \; 2$ respectively. By theorem 3.2.2,
$$c_n = (-1)^nA + (3)^n(Bn+C)$$
Using our initial conditions and stuff, we have:
$$c_n = 2(-1)^n + (3)^n(2n-1)$$
\item \textit{Find $b_n$ explicitly, where $b_n-5b_{n-1}+8b_{n-2}-4b_{n-3} = 0$ with initial conditions $b_0 = b_1 = 2, \; b_2 = 0$.}. We define out characteristic polynomial as
$$C(x) = x^3 - 5x^2 + 8x - 4 = (x-2)^2(x-1)$$
This polynomial has two roots: $1,\; 2$ with respective multiplicities $1,\; 2$. By theorem 3.2.2, our recurrence relation in closed form is written as
$$b_n = a + (bn + c)2^n$$
Solving for these unknown constants gives us the closed form
$$b_n = (-2)2^n  + n2^n$$
\end{enumerate} 
\end{enumerate}
%END%
\end{document}