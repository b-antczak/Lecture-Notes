\documentclass{report}
\usepackage[margin=1in, paperwidth=8.5in, paperheight=11in]{geometry}
%Math packages%
\usepackage{amssymb}
\usepackage{amsmath}
\usepackage{amsthm}
%Spacing%
\usepackage{setspace}
\onehalfspacing
%Lecture number%
\newcommand{\lectureNum}{10}
%Variables - Date and Course%
\newcommand{\curDate}{March 22, 2017}
\newcommand{\course}{MATH 239}
\newcommand{\instructor}{}
%Defining the example tag%
%\theoremstyle{definition}%
\newtheorem{ex}{Example}[section]
%Setting counter given the lecture number%
\setcounter{chapter}{\lectureNum{}}
%Package for drawing graphs%
\usepackage{tikz}
\usepackage{verbatim}
\usetikzlibrary{arrows}
\def\part#1{\item[\bf #1)]}

\begin{document}
%Note title%
\begin{center}
\begin{Large}
\textsc{\course{} | Tutorial \lectureNum{}}
\end{Large}
\end{center} 
\noindent \textit{Bartosz Antczak} \hfill
\textit{\curDate{}}
\rule{\textwidth}{0.4pt}
% Actual Notes%
\section*{Problem Set 2.8}
\begin{enumerate}
\item[3.] Write an unambiguous expression generating the the following binary strings:
\begin{enumerate}
\item \textit{$\{0,1\}-$strings with no substrings of 0s of length 3 and no substrings of 1s of length 2}
\subsubsection{Solution}
$$S = \{\varepsilon, 1\}\quad(\{0,00\}\{1\})^*\quad\{\varepsilon, 0, 00\}$$
\item \textit{$\{0,1\}-$strings with no 0 blocks of length 3 and no 1 blocks of length 2}
\subsubsection{Solution}
$$S = \{\varepsilon, 1, 111, 1111, \cdots\}\quad(\{0, 00, 0000, \cdots\}\{1, 111, 1111\})^*\quad\{\varepsilon, 0, 00, 0000, \cdots\}$$
\item \textit{The set of binary string in which 011 doesn't occur}
\subsubsection{Solution (using more star notation in this one)}
$$S = \{1\}^* \quad (\{0\}\{0\}^*\{1\})^*\quad \{0\}^*$$
\item Skipped this one because Prof. Postle thought it was too hard.
\item \textit{The set of binary strings where the 0 blocks have even length and the 1 blocks have odd length}
\subsubsection{Solution}
Other than maybe the first and last pieces, if we break up our strings into pieces at the end of each 0 block we get pieces of the form: $\{1, 111, 11111, \cdots\}\{00, 0000, \cdots\}$:
$$S = \{\varepsilon, 00, 0000, \cdots\} \quad (\{1, 111, 11111, \cdots\}\{00, 0000, \cdots\})^* \quad \{\varepsilon, 1, 111, 11111, \cdots\}$$
Simplified (by using more star notation), we get:
$$\{00\}^* \quad (\{1\}\{11\}^* \{00\}^*)^* \quad\{\varepsilon, 1, 111, 11111, \cdots\}$$
\item \textit{Set of all binary strings where}
\begin{itemize}
\item \textit{each odd length block of 0s is followed by a non-empty even length block of 1s,} \textbf{and}
\item \textit{each even length block of 0s is followed by an odd length block of 1s}
\end{itemize}
\subsubsection{Solution}
Keep in mind we don't consider blocks of length 0, because they're not blocks. So it's safe to assume that all of these blocks are non-empty.\\
We'll decompose our strings after blocks of 1s. With perhaps the exception of the first and last pieces, our pieces will look like:
\begin{itemize}
\item[i)] $\{0, 000, 00000, \cdots\}\{11, 1111, 111111, \cdots\} = \{0\}\{00\}^* \{11\}\{11\}*$
\item[ii)] $\{00, 0000, \cdots\}\{1, 111, 11111, \cdots\} = \{00\}\{00\}^*\{1\}\{11\}^*$
\end{itemize}
Our expression will look like:
$$S = \{1\}^* \quad {(\{0\}\{00\}^*\{11\}\{11\}^* \cup \{1\}\{11\}^*\{00\}\{00\}^*)}^*$$
\end{enumerate}
\item[5.]
\begin{enumerate}
\item Show that the generating series by length (i.e., the weight function is the length of the string) for binary strings in which every block of zeros has length $\geq 2$ and every block of ones has length $\geq 3$ is
$$\frac{(1-x-x^3)(1-x+x^2)}{1-2x+x^2-x^5}$$
\subsubsection{Solution}
Our expression for this set of binary strings is:
$$S = \{\varepsilon, 111, 1111, \cdots\} \quad {(\{00\}\{0\}^*\{111\}\{1\}^*)}^*\{\varepsilon, 00, 0000, \cdots\}$$
We define the generating series of each part of the expression (begin, mid, end):
\begin{align*}
\Phi_{begin}(x) &= 1 + x^3 + x^4 + \cdots = \frac{1}{1-x}-x-x^2\\
\Phi_{end}(x) &= \left(\frac{1}{1-x}\right)x^2\left(\frac{1}{1-x}\right)x^3 = y = \frac{1}{1-y}\frac{x^5}{(1-x)^2} \\
\Phi_{end}(x) &= 1 + x^2 + x^3 + \cdots  = \frac{1}{1-x}-x\\
\end{align*}
\end{enumerate} 
\end{enumerate}
%END%
\end{document}