\documentclass{report}
\usepackage[margin=1in, paperwidth=8.5in, paperheight=11in]{geometry}
%Math packages%
\usepackage{amsmath}
\usepackage{amssymb}
%\usepackage{MnSymbol}
\usepackage{amsthm}
%Spacing%
\usepackage{setspace}
\onehalfspacing
%Lecture number%
\newcommand{\lectureNum}{33}
%Variables - Date and Course%
\newcommand{\curDate}{March 29, 2017}
\newcommand{\course}{MATH 239}
\newcommand{\instructor}{Evelyne Smith-Roberge}
%Defining the example tag%
%\theoremstyle{definition}%
\newtheorem{ex}{Example}[section]
%Setting counter given the lecture number%
\setcounter{chapter}{\lectureNum{}}
%Package for drawing graphs%
\usepackage{tikz}
\usepackage{verbatim}
\usetikzlibrary{arrows}

\begin{document}
%Note title%
\begin{center}
\begin{Large}
\textsc{\course{} | Lecture \lectureNum{}}
\end{Large}
\end{center} 
\noindent \textit{Bartosz Antczak} \hfill
\textit{Instructor: \instructor{}} \hfill
\textit{\curDate{}}
\rule{\textwidth}{0.4pt}
%Actual Notes%
\onehalfspacing
\section{Repeated Roots in Recurrence Relations}
Last lecture we focused on recurrence relations, but their roots were distinct. Say we have $A(x) = \displaystyle \frac{1}{(1-2x)^2}$. Can we find an explicit expression for $a_n$? Yes we can! We simply solve
$$[x^n]\frac{1}{(1-2x)^2}$$
We solve this using our handy dandy negative binomial: $(1-y)^{-m} = \displaystyle\sum_{n \geq 0}{n + m - 1 \choose m-1}y^n$
\begin{align*}
[x^n](1-2x)^{-2} &= [x^n]\sum_{n \geq 0}{n+2-1 \choose 2-1}(2x)^n \\
&= [x^n]\sum_{n \geq 0}{n-1 \choose 1}2^nx^n\\
&= (n+1)2^n & \qed
\end{align*}
\begin{ex}
Solve the following recurrence relation: $a_n = 6a_{n-1} + 9a_{n-2}$, with $a_0 = a_1 = 1$
\end{ex}
\subsubsection{Solution to Example 33.1.1}
We define $Q(x) = 1 - 6x + 9x^2$. From this, our characteristic polynomial is $x^2 - 6x + 9 = (x-3)^2$.
\begin{align*}
a_n &= (An + B)3^n \\
a_0 = 1 &\implies B = 1 \\
a_1 = 1 &\implies A = -\frac{2}{3}
\end{align*}
Thus, $a_n$ in closed form is written as
$$a_n = \left(-\frac{2n}{3} + 1\right)3^n$$
\subsection{Theorem 1}
\begin{center}
\textit{Suppose ${(a_n)}_{n \geq 0}$ satisfies some recurrence relation $a_n + q_1a_{n-1} + \cdots + q_ka_{n-k}, \; n \geq k$.\\If the characteristic polynomial has roots $r_i$ with multiplicity $m_i, q \leq i \leq j$, the general solution is $$a_n = P_1(n)r_1^n + \cdots + P_j(n)r_j^n$$where $P_i(n)$ is a polynomial of degree less than $m_i$}
\end{center}\newpage
\begin{ex}
Find $a_n$ explicitly, where $a_n - 4a_{n-1} + 5a_{n-2}-2a_{n-3} = 0$ with initial conditions $a_0 = 1, a_1 = 1, a_2 = 2$.
\end{ex}
\subsubsection{Solution to Example 33.1.2}
\begin{itemize}
\item First step: find $Q(x)$ $$Q(x) = 1-4x+5x^2-2x^3$$
From this, our characteristic polynomial is: $$C(x) = x^3 - 4x^2 + 5x - 2 = (x-1)^2(x-2)$$
\item Next step: apply theorem 3.2.2. We'll have a solution of the form
$$a_n = (An+B)1^n + (c)2^n \qquad n \geq 0$$
\item Final step: refer to initial conditions to solve for $A$ and $B$:\\
\begin{align*}
a_0 = 1 &\implies 1 = B + c \\
a_1 = 1 &\implies 1 = A+B + 2c \\
a_2 = 2 &\implies 2 = 2A+B + 4c \\
\end{align*}
Solving these equations gives us: $A = -1,\; B = 0,\; C = 1$. Thus $$a_n = 2^n - n, \; n \geq 0$$
\end{itemize}
\begin{ex}
Find $c_n$ explicitly, where $c_n + 4c_{n-1} - 3c_{n-2} - 18c_{n-3} = 0, \; n \geq 3$ with initial conditions $c_0 = 0, c_1 = 2, c_2 = 13$.
\end{ex}
\subsubsection{Solution to Example 33.1.3}
Let $C(x)$ be our characteristic polynomial. Then
$$C(x) = x^3 + 4x^2 -3x - 18 = (x-2)(x+3)^2$$
This polynomial has roots $2, \; -3$ with multiplicity $1, \; 2$ respectively (this means that the root 2 will have only a constant next to it, and the root ($-3$) will have a linear polynomial next to it. We get from theorem 3.2.2 that $c_n$ will be of the form:
$$c_n = A\cdot 2^n + (B+cn)(-3)^n$$
From our initial conditions, we solve for the constants, which are $A = 1,\; B=-1, \; C=1$. To conclude, we solve
$$c_n = 2^n + (n-1)(-3)^n, \; n \geq 0$$
%END%
\end{document}