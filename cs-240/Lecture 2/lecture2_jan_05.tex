\documentclass{report}
\usepackage[margin=1in, paperwidth=8.5in, paperheight=11in]{geometry}
%Math packages%
\usepackage{amsmath}
\usepackage{amsthm}
%Spacing%
\usepackage{setspace}
\onehalfspacing
%Lecture number%
\newcommand{\lectureNum}{2}
%Variables - Date and Course%
\newcommand{\curDate}{January 5, 2017}
\newcommand{\course}{CS 240}
%Defining the example tag%
%\theoremstyle{definition}%
\newtheorem{ex}{Example}[section]
%Setting counter given the lecture number%
\setcounter{chapter}{\lectureNum{}}
%Package to insert code%
\usepackage{listings}
\usepackage{courier}
\usepackage{xcolor}
\lstset { 
    tabsize=2,
    breaklines=true,
    language=C++,
    backgroundcolor=\color{blue!8}, % set backgroundcolor
    basicstyle=\footnotesize\ttfamily,% basic font setting
}

\begin{document}
%Note title%
\begin{center}
\begin{Large}
\textsc{\course{} | Lecture \lectureNum{}}
\end{Large}
\end{center} 
\noindent \textit{Bartosz Antczak} \hfill
\textit{Instructor: Eric Schost} \hfill
\textit{\curDate{}}
\rule{\textwidth}{0.4pt}
% Actual Notes%
\section{More on order notation}
\subsection{Proofs with big-O notation}
Proving that $f(n) \in O(g(n))$ from \textit{first principles} means that you have to find one value of $c$ and $n_0$ such that $0 \leq f(n) \leq c \, g(n)$, $\forall\, n \geq n_0$.
Let's see some examples:
\begin{ex}
Prove that $2n^2 + 3n + 11 \in O(n^2)$. (To prove this, we use a similar approach to the example from last lecture).
\end{ex}
\noindent For $n \geq 1,$
\begin{align}
11 &\leq 11n^2 \\
3n &\leq 3n^2 \\
2n^2 &\leq 2n^2 \\
2n^2 + 3n + 11 &\leq 16n^2 && \text{adding (2.1), (2.2) and (2.3)}
\end{align}
Let $c = 16$ and $n_0 = 1$. By first principles, $2n^2 + 3n + 11 \in O(n^2)$.
%Example 2%
\begin{ex}
Prove that $(n+1)^5 \in O(n^5)$.
\end{ex}
\noindent For $n \geq 1$,
\begin{align}
n + 1 &\leq 2n \\
(n + 1)^5 &\leq (2n)^5 \\
(n + 1)^5 &\leq 32n^5
\end{align}
Let $c = 32$ and $n_0 = 1$. By first principles, $(n + 1)^5 \in O(n^5)$.
%Example 3%
\begin{ex}
Prove that $n^2 + n \log_2 (n) \in O(n^2)$.
\end{ex}
\noindent For $n \geq 1$,
\begin{align}
\log_2 (n) &\leq n \\
n \log_2 (n) &\leq n^2 && \text{multiply (2.8) by }n \\
n^2 &\leq n^2 \\
n^2 + n \log_2 (n) &\leq 2n^2
\end{align}
Let $c = 2$ and $n_0 = 1$. By first principles, $n^2 + n \log_2 (n) \in O(n^2)$. \\
Later on we'll prove big-O notation using other methods, but right now we're just learning the basics.

\subsection{Some rules with big-O notation}
\noindent Suppose $f(n) \geq 0$ and $g(n) \geq 0$.
\begin{itemize}
\item[1)] If $a > 0$, then $f(n) \in O(a\, f(n))^*$ and $a \, f(n) \in O(f(n))^{**}$
\item[2)] If $f(n) \in O(g(n))$ and $g(n) \in O(h(n))$, then $f(n) \in O(h(n))^{***}$
\end{itemize}
%roof 1%
\textbf{Proof of (*):}
\begin{align}
0 \leq f(n) \leq \frac{1}{a} \cdot a \cdot f(n) \qquad \forall\, n \geq 1
\end{align}
We let $c = \frac{1}{a}$ and $n_0 = 1$, and using first principles we have proven that $f(n) \in O(a\, f(n))$.\\\\
%Proof 2
\textbf{Proof of (**):}
\begin{align}
0 \leq a\, f(n) \leq a \, f(n) \qquad \forall\, n \geq 1
\end{align}
We let $c = a$ and $n_0 = 1$, and using first principles we have proven that $a\, f(n) \in O(f(n))$.\\\\
\textbf{Proof of (***):}\\
By our assumption, there exists $c_1$ and $n_1$ such that
\begin{equation}
0 \leq f(n) \leq c_1 \, g(n) \qquad n \geq n_1
\end{equation}
and also there exists $c_2$ and $n_2$ such that
\begin{equation}
0 \leq g(n) \leq c_2 \, h(n) \qquad n \geq n_2
\end{equation}
Now, if $n \geq n_1$ and $n \geq n_2$
\begin{equation}
0 \leq f(n) \leq c_1 \, g(n) \leq c_1\,c_2\,h(n)
\end{equation}
We let $c = c_1c_2$ and $n_0 =$ max$(n_1,\,n_2)$, and using first principles we have proven $f(n) \in O(h(n))$.

\subsection{Proofs with big-Omega notation}
Big-Omega notation is the reverse of big-O notation (rather than $f$ being at most $O(n)$, $f$ is at least $\Omega(n)$). To prove that $f(n) \in \Omega(g(n))$, we have to find one value $c$ and one integer $n_0$ such that $0 \leq c \, g(n) \leq f(n)$ (this is first principles).
\begin{ex}
Prove that $n^3 \log_2(n) \in \Omega(n^3)$.
\end{ex}
\noindent For all $n \geq 2$
\begin{align}
\log_2(n) &\geq 1 \\
n^3 \log_2(n) &\geq n^3 && \text{multiply 2.17 by }n^3
\end{align}
We take $c=1$ and $n_0 = 2$. By first principles we have proven $n^3 \log_2(n) \in \Omega(n^3)$.

\subsection{Proofs with big-Theta notation}
To prove this, we use the same approach as we did with big-O and big-Omega notation.

\subsection{Little-O notation}
$f(n) \in o(g(n))$: this notation is similar to big-O notation, but we use this when we want to say that $f(n)$ is \textit{much less} than $o(g(n))$ (rather than less than or equal to $O(n)$).\\
To prove that $f(n) \in o(g(n))$, we are given $c > 0$, and you have to find $n_0$ such that $0 \leq f(n) < c\,g(n)$, $\forall\, n > n_0$.
\begin{ex}
Prove that $n \in o(n^2)$, given that $c > 0$
\end{ex}
\noindent We have to find $n_0$ such that
\begin{align}
n &< c\,n^2 \qquad n \geq n_0 \\
\iff 1 &< c\,n \\
\iff \frac{1}{c} &< n
\end{align}
We take $n_0 = \frac{1}{c}$ and using first principles, we have proven $n \in o(n^2)$.

\begin{ex}
Prove that $2010n^2 + 1388n \in o(n^3)$, given that $c > 0$
\end{ex}
\noindent We have to find $n_0$ such that $2010n^2 + 1388n < c\,n^3$ for $n \geq n_0$.\\For $n \geq 1,$
\begin{align}
n &\leq n^2 \\
1388n &< 1388n^2 \\
\implies 2010n^2 + 1388n &< 3398\,n^2 < 4000n^2
\end{align}
To finish this proof, we just have to find $n_0$ such that 4000$n^2 < c\,n^3$ for $n \geq n_0$
\begin{equation}
4000n^2 < c\,n^3 \iff 4000 < c\,n \iff \frac{4000}{c} < n
\end{equation}
We take $n_0 = \frac{4000}{c}$ and using first principles, we have proven $2010n^2 + 1388n \in o(n^3)$.

\subsection{Comparing big-O and little-O}
\begin{ex}
$O(1)$ and $o(1)$
\end{ex}
\begin{itemize}
\item[1)] $f(n) \in O(1)$ means that $f$ is bounded and that there exists a constant $M$ such that $0 \leq f(n) \leq M$
\item[2)] $f(n) \in o(1)$ means that $\displaystyle \lim_{n \to \infty} f(n) = 0$
\end{itemize}
\newpage
\section{Complexity of Algorithms}
Let $T_A(I)$ denote the running time of an algorithm $A$ on instance $I$. We consider two cases for $T$:
\begin{itemize}
\item \textbf{Average-case}: the average running time of $A$ over all instances of size $n$
\item \textbf{Worst-case}: the longest running time of $A$ over all instances of size $n$
\end{itemize}
%END%
\end{document}