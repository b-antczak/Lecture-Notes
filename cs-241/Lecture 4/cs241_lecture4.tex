\documentclass{report}
\usepackage[margin=1in, paperwidth=8.5in, paperheight=11in]{geometry}
%Math packages%
\usepackage{amsmath}
\usepackage{amsthm}
%Spacing%
\usepackage{setspace}
\onehalfspacing
%Lecture number%
\newcommand{\lectureNum}{4}
%Variables - Date and Course%
\newcommand{\curDate}{January 12, 2017}
\newcommand{\course}{CS 241}
\newcommand{\instructor}{Kevin Lanctot}
%Defining the example tag%
%\theoremstyle{definition}%
\newtheorem{ex}{Example}[section]
%Setting counter given the lecture number%
\setcounter{chapter}{\lectureNum{}}
%Package to insert code%
\usepackage{listings}
\usepackage{courier}
\usepackage{xcolor}
\lstset { %
    tabsize=2,
    breaklines=true,
    language=C++,
    backgroundcolor=\color{blue!8}, % set backgroundcolor
    basicstyle=\footnotesize\ttfamily,% basic font setting
}
\begin{document}
%Note title%
\begin{center}
\begin{Large}
\textsc{\course{} | Lecture \lectureNum{}}
\end{Large}
\end{center} 
\noindent \textit{Bartosz Antczak} \hfill
\textit{Instructor: \instructor{}} \hfill
\textit{\curDate{}}
\rule{\textwidth}{0.4pt}
% Actual Notes%
\subsection{Today's lecture plan}
\begin{itemize}
\item Labels
\item Arrays
\item Structure of Assembly files
\item Output
\item Function calls
\end{itemize} 
\section{Additional MIPS instructions}
\subsection{Branch Labels in MIPS}
Rather than including the immediate variable in a branching instruction, we can use labels as a substitute for calculating the offset (how many instructions we're skipping). For example:\\
\textit{Without label (what we're used to):}
\begin{itemize}
\item[\textbf{0.}] \texttt{mult \$2, \$3}
\item[\textbf{4.}] \texttt{mflo \$2}
\item[\textbf{8.}] \texttt{sub \$3, \$3, \$1}
\item[\textbf{12.}] \texttt{bne \$3, \$0, \underline{-4}}
\end{itemize}
\textit{With a label (we called it ``loop"):}
\begin{itemize}
\item[\textbf{0.}] \texttt{\underline{loop:} mult \$2, \$3}
\item[\textbf{4.}] \texttt{mflo \$2}
\item[\textbf{8.}] \texttt{sub \$3, \$3, \$1}
\item[\textbf{12.}] \texttt{bne \$3, \$0, \underline{loop}}
\end{itemize}
Here, loop is the label. It's placed in the first column and it ends with a colon. As shown, it replaces the immediate value that determines how many instructions to skip.
\subsubsection{Labels and Scope}
Labels can be created manually and automatically. They must also be unique within scope.
\begin{ex}
Another example of using a label:
\end{ex}
\begin{itemize}
\item[\textbf{0.}] \texttt{beq \$1, \$0, useGST}
\item[\textbf{4.}] \texttt{add \$2, \$2, \$4}
\item[\textbf{8.}] \texttt{beq \$0, \$0, final}
\item[\textbf{12.}] \texttt{useGST: add \$2, \$2, \$3}
\item[\textbf{16.}] \texttt{final: }\textit{; continue with program}
\end{itemize}
\subsection{What's in an Assembly File?}
\begin{itemize}
\item Assembly instructions
\item Label declarations
\item Data definitions (such as \texttt{.word})
\item Comments (start with a semi-colon)
\end{itemize}
A suggested format for an assembly file is to divide your data into three columns: label, instructions, and comments:
\begin{center}
\begin{tabular}{ l l l }
\textbf{Labels:} & \textbf{Instructions} & \textbf{Comments} \\
\texttt{loop:} & \texttt{add \$2, \$2, \$4} & \textit{; \$2 = 32 in decimal, \$4 is zero}
\end{tabular}
\end{center}
\subsection{Arrays in MIPS}
\underline{Recall}: to access a particular element in an array, we need to know the element's address in the array. The size of each element in the array is 4 bytes.\\
The \textit{base address} of the array is the address of the first element (0th index in the array). To access each successive element, add 4 to the current element's address. For instance, if the base address is in register \$1, then the $i$-th element has an address of \$1 + 4$i$.
\subsection{Output}
Input and output from devices are treated just like reading from and writing to memory, which means that we'll use the MIPS instructions \texttt{lw} and \texttt{sw}; however, to specify that we want to input or output data, we'll load and store data to specific locations in memory.\\
In CS 241, to output a char to the screen, store the ASCII value of the character to memory address FFFF000C$_{\mathrm{hex}}$ (note that this address is fixed, but in reality this address is different for various devices). This is also called the \textit{video output}. This method outputs one character at a time. \newpage
\begin{ex}
Print ``C" on the screen
\end{ex}
\begin{tabular}{ l l }
\texttt{lis \$1} & \textit{; address of output buffer} \\
\texttt{.word 0xFFFF000C} & \\
\texttt{lis \$2} & \textit{; ASCII C} \\
\texttt{.word 67} & \\
\texttt{sw \$2, 0(\$1)} & \textit{; write to screen}
\end{tabular}

\subsection{Writing Functions in MIPS}
In MIPS, we don't have functions; instead, we have \textit{subroutines}. We will implement \texttt{jr}, \texttt{jalr}, labels, and registers to \underline{implement} functions (i.e., we have to build them).
\subsubsection{Subroutines vs. Functions}
We define
\begin{itemize}
\item Labels as function names
\item Arguments and return values as values stored in certain registers or a specific memory location
\end{itemize}
Since subroutines can access any register they want, there is no local scope.\\
Some challenges using subroutines:
\begin{itemize}
\item The subroutine call is static (since the address of it is always in the same location)
\item The return is dynamic, because the location of the function call could happen anywhere (consider a C++ program, you can call a function from anywhere)
\item Nested calls, recursion
\end{itemize}
To navigate through the code, we'll use two instructions:
\begin{itemize}
\item \texttt{jalr \$s}: means \textit{jump and link register}. It behaves just like \texttt{jr} (i.e., sets the PC to the address stored in \$s) with an additional action: it also copies the address of the next instruction to \$31 (this instruction is used to jump into a function)
\item \texttt{jr \$s}: already defined from previous lecture. This instruction is used to exit a function
\end{itemize}
Where do we store essential data in the registers? Since registers are not local variables, any function can access them. What if we have important data in \$2, but a certain function accesses \$2 and changes the value. Where do we store this essential data before we call that function? We store it in a \textit{stack}.
\subsubsection{The Run-time Stack}
We store the stack in a part of memory (i.e., RAM). It follows a last-in, first-out queue. Some things to note about the stack:
\begin{itemize}
\item The stack grows downward in memory. This means that in order to access the $i$th item in the stack, we subtract 4$i$ from the base address. For instance, the address of the first item below the base address is (base address$- 4$)
\item The address of the bottom of the stack is stored in the \textit{stack pointer} (SP) register. In our MIPS simulator, we use \$30 (however typically it's \$29)
\end{itemize}
\begin{center}
| At this moment, you can do 7 out of 8 questions on assignment 2 |
\end{center}
%END%
\end{document}