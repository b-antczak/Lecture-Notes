\documentclass{report}
\usepackage[margin=1in, paperwidth=8.5in, paperheight=11in]{geometry}
%Math packages%
\usepackage{amsmath}
\usepackage{amsthm}
\usepackage{amssymb}
%Spacing%
\usepackage{setspace}
\onehalfspacing
%Lecture number%
\newcommand{\lectureNum}{12}
%Variables - Date and Course%
\newcommand{\curDate}{May 28, 2017}
\newcommand{\course}{AMATH 250}
\newcommand{\instructor}{Zoran Miskovic}
%Defining the example tag%
%\theoremstyle{definition}%
\newtheorem{ex}{Example}[section]
%Setting counter given the lecture number%
\setcounter{chapter}{\lectureNum{}}
\usepackage{graphicx}
\usepackage{calc}
\usepackage{listings}

\lstset{frame=tb,
  aboveskip=3mm,
  belowskip=3mm,
  showstringspaces=false,
  columns=flexible,
  basicstyle={\small\ttfamily},
  numbers=none,
  breakatwhitespace=true,
  tabsize=3
}


\begin{document}
%Note title%
\begin{center}
\begin{Large}
\textsc{\course{} | Lecture \lectureNum{}}
\end{Large}
\end{center} 
\noindent \textit{Bartosz Antczak} \hfill
\textit{Instructor: \instructor{}} \hfill
\textit{\curDate{}}
\rule{\textwidth}{0.4pt}
% Actual Notes%
\section{Finishing up Buckingham Pi Theorem}
\begin{ex}
Throwing a ball vertically in the air
\end{ex}\noindent
We list our parameters and their respective dimensions
$$\begin{matrix}
& h & t & v_0 & g & m  \\
M & 0 & 0 & 0 & 0 & 1 \\
L & 1 & 0 & 1 & 1 & 0 \\
T & 0 & 1 & -1 & -2 & 0
\end{matrix}$$
Here, we have $P = 5-3 = 2$ dimensionless parameters.
What happens if we forget to list $m$? In that case, we'll have $N = 4$ and the rank of the new matrix would be $n=2$ (one row of 0's). So in this case, we'd still have $p = 4-2 = 2$ parameters, but this is risky, so don't try it. \\
What happens if we forget $t$? It's totally fine to leave $t$ out as well. Here's why: we use conservation of energy to solve for $v_0$ given $h_{max}$. Initially, we have $h(0) = 0$ and $v(0) = v_0$. At this point in time, all of our energy is in the form of kinetic energy: $\frac{1}{2}mv_0^2$. \\
At the max height, we have $v = 0$ and $h = h_{max}$. Here, all of our energy is converted to potential energy: $mgh_{max}$, and so our solution is
$$\frac{1}{2}mv_0^2 = mgh_{max} \implies h_{max} = \frac{v_0^2}{2g}$$
By Newton's law
\begin{equation}
\frac{d^2h}{dt^2} = -g = \frac{dv}{dt}
\end{equation}
To find a DE relating $v$ and $h$, assume $v = v(h(t))$. By Chain Rule
\begin{equation}
\frac{dv}{dt} = \frac{dv}{dh}\cdot \frac{dh}{dt} = v\cdot \frac{dv}{dh}
\end{equation}
Sub (12.2) into (12.1)
\begin{align}
v\cdot \frac{dv}{dh} &= - g \\
\int v \; dv &= -\int g \; dh \\
\frac{1}{2}mv^2 &= -mgh + d && \text{(Multiply by }m)\\
\frac{1}{2}mv^2 + mgh &= d
\end{align}
This equation is independent of time, ad it states that energy difference is constant (i.e., energy is conserved).
\subsection{What does Buckingham Pi Theorem tell us?}
$$\begin{matrix}
& h  & v_0 & g & m  \\
M & 0 & 0 & 0 & 1 \\
L & 1 & 1 & 1 & 0 \\
T & 0 & -1 & -2 & 0
\end{matrix}$$
We have $N = 4$ and $r = 3$, and so we have one dimensionless product $\Pi$.
\begin{equation}
\Pi = h_{max}^a\cdot v_0^c\cdot g^d\cdot m^e
\end{equation}
Solving DP = 0, we have
\begin{align*}
e &= 0 \\
a + c + d &= 0 \\
-c -2d &= 0
\end{align*}
Choosing $d$ as an arbitrary constant, we have
$$h_{max}^d \cdot v_0^{-2d} \cdot g^d$$
Let $d=1$
$$\Pi = \frac{h_{max}g}{v_0^2} = \text{const}$$
This solution gives us the conversation of energy
$$\frac{1}{2}v^2 = gh_{max} \implies \frac{1}{2} = \frac{gh_{max}}{v^2}$$
\section{2nd order DEs (3.1.1, 3.1.3)}
The general form of a 2nd order DE is
$$\frac{d^2y}{dx^2} = f\left(x,y,\frac{dy}{dx}\right)$$
Initial value problems (IVPs) for this equation contain \textbf{two} initial conditions at some point $x_0$. Given $y_0$ and $v_0$,
\begin{align*}
y(x_0) &= y_0 \\
\frac{dy}{dx}(x_0) &= v_0
\end{align*}
This means that at point $(x_0, y_0)$, the slope is $v_0$.\\
We'll study linear 2nd order DEs. The general form is
$$\frac{d^2y}{dx^2} + P(x)\frac{dx}{dy} + Q(x)y = F(x)$$ 
where $P, Q, F$ are given functions of $x$. If $F(x) = 0$ , then our formula is homogeneous. If $P$ and $Q$ are constant on some interval $I$, then our formula has constant coefficients. \\
The Existence and Uniqueness Theorem for the solutions of the IVP states that if $P,Q, F$ are all continuous on some interval $I$, then there exists a unique solution of the general DE.
\subsection{Mechanical Oscillator}
Consider a spring-loaded mass (just like grade 12 physics example). Let $y(t)$ be the displacement of $m$ from its equilibrium position. Newton's law is
$$m\frac{d^2y}{dt^2} = F_{total}$$
We also define
\begin{align*}
F_d &= -\alpha \frac{dy}{dt} \\
F_r &= ky \qquad (k=\text{spring constant)} \\
F_{ext}(t) &= F(t)
\end{align*}
where $F(t)$ is constant. From this we define a 2nd order DE
\begin{align*}
m\frac{d^2y}{dx^2} &= -\alpha \frac{dy}{dt} - ky + F_{ext} \\
F_{ext}  &= m\frac{d^2y}{dx^2} + \frac{dy}{dt} + ky
\end{align*}
%END%
\end{document}