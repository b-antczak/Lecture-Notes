\documentclass{report}
\usepackage[margin=1in, paperwidth=8.5in, paperheight=11in]{geometry}
%Math packages%
\usepackage{amsmath}
\usepackage{amsthm}
\usepackage{amssymb}
%Spacing%
\usepackage{setspace}
\onehalfspacing
%Lecture number%
\newcommand{\lectureNum}{3}
%Variables - Date and Course%
\newcommand{\curDate}{May 7, 2017}
\newcommand{\course}{AMATH 250}
\newcommand{\instructor}{Zoran Miskovic}
%Defining the example tag%
%\theoremstyle{definition}%
\newtheorem{ex}{Example}[section]
%Setting counter given the lecture number%
\setcounter{chapter}{\lectureNum{}}
\usepackage{graphicx}
\usepackage{calc}
\usepackage{listings}

\lstset{frame=tb,
  aboveskip=3mm,
  belowskip=3mm,
  showstringspaces=false,
  columns=flexible,
  basicstyle={\small\ttfamily},
  numbers=none,
  breakatwhitespace=true,
  tabsize=3
}


\begin{document}
%Note title%
\begin{center}
\begin{Large}
\textsc{\course{} | Lecture \lectureNum{}}
\end{Large}
\end{center} 
\noindent \textit{Bartosz Antczak} \hfill
\textit{Instructor: \instructor{}} \hfill
\textit{\curDate{}}
\rule{\textwidth}{0.4pt}
% Actual Notes%
\subsubsection{Last Time}
1st order DEs and seperable DEs of the form
$$\frac{dy}{dx} = A(x) B(y)$$
We looked at two examples:
\begin{equation}
    \frac{dy}{dx} = x\sqrt{y} \implies y = \left(\frac{x^2}{y} + c\right) \qquad c \in \mathbb{R}
\end{equation}
\begin{equation}
    \frac{dy}{dx} = -\frac{x}{y} \implies y = x^2 + y^2 = d \qquad d \in \mathbb{R}
\end{equation}
\section{Qualitative Sketch of a Solution (1.2.4)}
When sketching DEs, we focus on just drawing certain graphs (since there may be infinitely many) at the ``breakpoints".
\begin{ex}
Draw a qualitative sketch of equation 3.1
\end{ex}
Let's consider some cases:
\begin{enumerate}
    \item $\frac{dy}{dx} = 0 \implies x = 0 \text{ or } y = 0$
        
    \vspace{0.0cm}
    \textbf{Solution:} draw the lines $x = 0, y = 0$
    \item $c = 0$
        
    \vspace{0.0cm}
    \textbf{Solution:} draw the graph $y = \frac{x^4}{16}$
    \item $c > 0 \text { or } c < 0$
        
    \vspace{0.0cm}
    \textbf{Solution:} pick some arbitrary value such as $c = 1$ or $c = -2$ and plot that graph
\end{enumerate}
The qualitative sketch of equation 3.2 is trivial and therefore skipped.
\section{Solving Linear DEs: integrating factor method (1.2.3)}
This method follows a series of steps:
\begin{enumerate}
    \item Set DE to standard form:
    $$ \frac{dy}{dx} + k(x)y = f(x)$$
    where $k$ and $f$ are given functions.
    \item Define the following integrating factor:
    $$I(x) = e^{\int k(x) \; dx}$$
    Also notice that
    $$\frac{dI}{dx} = e^{\int k(x) \; dx} \cdot \frac{d}{dx}\int k(x) \;dx = I(x) k(x)$$
    \item Multiply the DE from step 1 by $I(x)$ and simplify:
    \begin{align}
        I(x) \frac{dy}{dx} + I(x)k(x)y &= I(x)f(x) \\
        I(x)\frac{dy}{dx} + \frac{dI}{dx}y &= I(x)f(x) \\
        \frac{d}{dx}\left[I(x)y\right] &= I(x)f(x) \\
        I(x)y &= \int I(x)f(x) \; dx
    \end{align}
    \item Solve (3.6) for $y(x)$.
\end{enumerate}
\begin{ex}
Find the general solution of $x\frac{dy}{dx} = x^2 + 2y$
\end{ex}
\begin{align*}
x\frac{dy}{dx} = x^2 + 2y \\
x\frac{dy}{dx} - \frac{2y}{x} = x  && \text{(Divide by }x \neq 0)
\end{align*}
Let $k(x) = -\frac{2}{x}$ and $f(x) = x$. We have:
$$
\int k(x) \; dx = -2 \int \frac{dx}{x} = -\ln x^2 = \ln \left(\frac{1}{x^2}\right)
$$
This means that we set our integrating factor to
$$I(x) = e^{\ln\left(\frac{1}{x^2}\right)} = \frac{1}{x^2}$$
And so our solution is:
\begin{align*}
    \frac{d}{dx}\left(\frac{y}{x^2}\right) &= \frac{1}{x} \\
    \frac{y}{x^2} &= \ln |x| \\
    y &= (x^2 \ln |x|) + c
\end{align*}
\subsubsection{Quick note}
We did not add a constant when solving
$$
\int k(x) \; dx = \ln \left(\frac{1}{x^2}\right)
$$
This is okay, however, because this constant will get cancelled out later on in our calculations.
%END%
\end{document}
