\documentclass{report}
\usepackage[margin=1in, paperwidth=8.5in, paperheight=11in]{geometry}
%Math packages%
\usepackage{amsmath}
\usepackage{amsthm}
\usepackage{amssymb}
%Spacing%
\usepackage{setspace}
\onehalfspacing
%Lecture number%
\newcommand{\lectureNum}{3}
%Variables - Date and Course%
\newcommand{\curDate}{May 25, 2017}
\newcommand{\course}{AMATH 250}
\newcommand{\instructor}{}
%Defining the example tag%
%\theoremstyle{definition}%
\newtheorem{ex}{Example}[section]
%Setting counter given the lecture number%
\setcounter{chapter}{\lectureNum{}}
\usepackage{graphicx}
\usepackage{calc}
\usepackage{listings}

\lstset{frame=tb,
  aboveskip=3mm,
  belowskip=3mm,
  showstringspaces=false,
  columns=flexible,
  basicstyle={\small\ttfamily},
  numbers=none,
  breakatwhitespace=true,
  tabsize=3
}


\begin{document}
%Note title%
\begin{center}
\begin{Large}
\textsc{\course{} | Tutorial \lectureNum{}}
\end{Large}
\end{center} 
\noindent \textit{Bartosz Antczak} \hfill
\textit{\curDate{}}
\rule{\textwidth}{0.4pt}
% Actual Notes%
\section*{Problem 1}
Make the following DE dimensionless. This DE models the velocity of a ball thrown upward with initial velocity $v_{init}$ and $R$ is radius of earth.
\begin{equation}
v\frac{dv}{dr} = -\frac{gR^2}{r^2}
\end{equation}
\begin{enumerate}
\item[(a)] Define characteristic variables
$$v_c = \sqrt{Rg}$$
$$r_c = R$$
\item[(b)] Define dimensionless variables
$$\mathcal{V} = \frac{v}{v_c} = \frac{v}{\sqrt{Rg}} \implies dV = \frac{dv}{\sqrt{Rg}}$$
$$\mathcal{R} = \frac{r}{r_c} = \frac{r}{R} \implies d\mathcal{R} = \frac{dr}{R}$$
\item[(c)]
\begin{align}
\frac{dv}{dr}&=\frac{dv}{d\mathcal{V}}\cdot\frac{d\mathcal{{V}}}{d\mathcal{R}}\cdot\frac{d\mathcal{R}}{dr} \\
&= \sqrt{Rg}\cdot\frac{d\mathcal{V}}{d\mathcal{R}}\cdot\frac{1}{R} \\
&= \sqrt{\frac{g}{R}}\cdot \frac{d\mathcal{V}}{d\mathcal{R}}
\end{align}
\item[(d)] Sub int original DE
\begin{align}
\mathcal{V}\sqrt{Rg}\sqrt{\frac{g}{R}} \cdot\frac{d\mathcal{V}}{d\mathcal{R}} &= -\frac{g}{\mathcal{R}^2} \\
\mathcal{V}\cdot \frac{d\mathcal{V}}{d\mathcal{R}} &= -\frac{1}{\mathcal{R}^2}
\end{align}
Now we solve our dimensionless DE. We see that $v(R) = v_{init}$. Now $r = R\mathcal{R}$, and letting $r = R$ (at Earth surface level), we have $\mathcal{R} = 1$. So we have on IC
$$\mathcal{V}(1) = \frac{v_{init}}{\sqrt{Rg}}$$
So now we solve
\begin{align}
\int \mathcal{V} d\mathcal{V} &= - \int \mathcal{R}^2 d\mathcal{R} \\
\frac{1}{2}\mathcal{V}^2 &= \mathcal{R}^{-1} + C \\
\mathcal{V} &= \sqrt{\frac{2}{\mathcal{R}}+C}
\end{align}
Plugging in our DE, we have
$$\mathcal{V}(\mathcal{R}) = \sqrt{\frac{2}{R} + \frac{v_{init}^2}{Rg} - 2}$$
\item[(e)] Now re-dimensonalize the DE. Solve for $v(r)$. We do this by subbing in the values for $\mathcal{V}$ and $\mathcal{R}$.
\begin{align}
\frac{v}{\sqrt{Rg}} &= \sqrt{\frac{2R}{r} + \frac{v_{init}^2}{Rg} - 2} \\
v(r) &= \sqrt{\frac{2R^2g}{r} + v_{init}^2 - 2Rg} \qquad \qed
\end{align}
\end{enumerate}
%END%
\end{document}