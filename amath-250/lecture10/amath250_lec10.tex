\documentclass{report}
\usepackage[margin=1in, paperwidth=8.5in, paperheight=11in]{geometry}
%Math packages%
\usepackage{amsmath}
\usepackage{amsthm}
\usepackage{amssymb}
%Spacing%
\usepackage{setspace}
\onehalfspacing
%Lecture number%
\newcommand{\lectureNum}{10}
%Variables - Date and Course%
\newcommand{\curDate}{May 23, 2017}
\newcommand{\course}{AMATH 250}
\newcommand{\instructor}{Zoran Miskovic}
%Defining the example tag%
%\theoremstyle{definition}%
\newtheorem{ex}{Example}[section]
%Setting counter given the lecture number%
\setcounter{chapter}{\lectureNum{}}
\usepackage{graphicx}
\usepackage{calc}
\usepackage{listings}

\lstset{frame=tb,
  aboveskip=3mm,
  belowskip=3mm,
  showstringspaces=false,
  columns=flexible,
  basicstyle={\small\ttfamily},
  numbers=none,
  breakatwhitespace=true,
  tabsize=3
}


\begin{document}
%Note title%
\begin{center}
\begin{Large}
\textsc{\course{} | Lecture \lectureNum{}}
\end{Large}
\end{center} 
\noindent \textit{Bartosz Antczak} \hfill
\textit{Instructor: \instructor{}} \hfill
\textit{\curDate{}}
\rule{\textwidth}{0.4pt}
% Actual Notes%
\subsubsection{Last Time}
Dimensionless forms of DEs. We looked at two examples
\begin{enumerate}
\item Salt concentration$$\frac{dm}{dt} = \frac{f}{V}m + fc_{in}$$
We converted that DE into 
$$\frac{d\mathcal{M}}{dt} + \mathcal{M} = 1$$
\item Skydiver
$$\frac{dv}{dt} = mg - \alpha g \implies \frac{dV}{dt} + V = 1$$
\end{enumerate}
\section{More on dimensionless forms of DEs}
There is a third example that we'll cover on this topic: Newton's heating/cooling law. Define $T(t)$ as the temperature of the heating/cooling body. Define $T_A$ as the ambient temperature. We have
\begin{equation}
\frac{dT}{dt} = -k(T - T_A)
\end{equation}
We identify
\begin{itemize}
\item Characteristic time: $t_c = \frac{1}{k}$
\item Characteristic temperature: $T_c = T_A$
\end{itemize}
Define
\begin{itemize}
\item Dimensionless time: $\tau = \frac{t}{t_c}$
\item Dimensionless temperature: $\theta = \frac{T}{T_c}$
\end{itemize}
Sub our defined variables into (10.1) and we get
$$\frac{d\theta}{d\tau} + \theta = 1$$
\section{Deducing physical relations using dimensionless analysis via Buckingham Pi Theorem [2.2.3] ($B\Pi T$)}
\begin{ex}
Skydiver problem revisited
\end{ex}
For our defined DE for this problem, we have a relation between 4 physical quantities
\begin{equation}
F(v,m,g,\alpha) = 0 
\end{equation}
$B\Pi T$ tells us that (10.2) must be equivalent to a relation of the form
\begin{equation}
f(\Pi_1, \Pi_2, \ldots) = c \qquad \exists c \in \mathrm{R}
\end{equation}
where $\Pi_1, \Pi_2, \ldots$ are all possible independent dimensionless products of quantities $v,m,g,\alpha$.\\
How do we find all possible products? First, we write the most general product of $\{v,m,g,\alpha\}$
\begin{equation}
\Pi = v^a m^bg^c\alpha^d
\end{equation}
with unknown values $a,b,c,d$.\\
From there, we define the dimensions of $\Pi$ as
\begin{align}
[\Pi] &= [v]^a [m]^b [g]^c[\alpha]^d \\
&= L^{a+c}\cdot T^{-(a+2c+\alpha)}\cdot M^{b+d}
\end{align}
To make $\Pi$ dimensionless, we need to cancel out the dimensions. This means we have a system of equations
\begin{align*}
a + C &= 0 \\
-a - 2c - d &= 0 \\
b+d &= 0
\end{align*}
We see that there are actually an infinite number of solutions. We want to solve this system for $a,c,d$ in terms of $b$, which is just some arbitrary real number. Solving, we have
\begin{align*}
a&= -b \\
c &= b \\
d &= -b
\end{align*}
Subbing in those values into (10.4) results
\begin{align}
\Pi &= v^{-b}m^bg^b\alpha^{-b} \\
&= \left(\frac{mg}{v\alpha}\right)^b
\end{align}
Since $b$ is arbitrary, let $b=1$, so that $\Pi = \frac{mg}{v\alpha}$. by the $B\Pi T$, we must have some function $f$
$$f(\Pi) = c \implies f\left(\frac{mg}{v\alpha}\right) = c$$
If $f$ is invertible, we can write $\frac{mg}{v\alpha} = f^{-1} = k$. From this, we conclude that the terminal velocity is $v = \frac{1}{k}\frac{mg}{\alpha}$.
\section{Complete sets of dimensionless quantities}
In general, if we have $N$ physical quantities $Q_1, Q_2, \ldots, Q_N$ and the rank $r$ of the system, then there will be $P = N-r$ independent dimensionless products $\Pi_1, \Pi_2, \ldots, \Pi_P$. Then the relation $F(Q_1, Q_2, \ldots, Q_N) = 0$ can be expressed as $f(\Pi_1, \Pi_2, \ldots, \Pi_P) = c$. Let's refer to our system in the previous example:
$$
M= DP =
  \begin{bmatrix}
    1 & 0 & 1 & 0 \\
    -1 & 0 & -2 & -1 \\
    0 & 1 & 0 & 1
  \end{bmatrix}
    \begin{bmatrix}
    a\\b\\c\\d
  \end{bmatrix}
  =
    \begin{bmatrix}
    0\\0\\0
  \end{bmatrix}
$$
It is easy to write $D$ by inspection:
\begin{center}

\begin{tabular}{c|cccc}
 & $v$&$m$&$g$&$\alpha$\\\hline
 $L$ & 1 & 0 & 1 & 0 \\
   $T$ & $-1$ & 0 & $-2$ & $-1$ \\
  $M$ &  0 & 1 & 0 & 1
\end{tabular}

\end{center}
%END%
\end{document}