\documentclass{report}
\usepackage[margin=1in, paperwidth=8.5in, paperheight=11in]{geometry}
%Math packages%
\usepackage{amsmath}
\usepackage{amsthm}
\usepackage{amssymb}
%Spacing%
\usepackage{setspace}
\onehalfspacing
%Lecture number%
\newcommand{\lectureNum}{2}
%Variables - Date and Course%
\newcommand{\curDate}{May 18, 2017}
\newcommand{\course}{AMATH 250}
\newcommand{\instructor}{}
%Defining the example tag%
%\theoremstyle{definition}%
\newtheorem{ex}{Example}[section]
%Setting counter given the lecture number%
\setcounter{chapter}{\lectureNum{}}
\usepackage{graphicx}
\usepackage{calc}
\usepackage{listings}

\lstset{frame=tb,
  aboveskip=3mm,
  belowskip=3mm,
  showstringspaces=false,
  columns=flexible,
  basicstyle={\small\ttfamily},
  numbers=none,
  breakatwhitespace=true,
  tabsize=3
}


\begin{document}
%Note title%
\begin{center}
\begin{Large}
\textsc{\course{} | Tutorial \lectureNum{}}
\end{Large}
\end{center} 
\noindent \textit{Bartosz Antczak} \hfill
\textit{\curDate{}}
\rule{\textwidth}{0.4pt}
% Actual Notes%
\section{Application of DEs}
\begin{ex}
Consider a pond that initially contains 10 million gallons of fresh water. Water containing an undesirable chemical flows into the pond at the rate of 5 million gal/year, and the mixture in the pond flows out at the same rate. The concentration $\gamma(t)$ of chemical in the incoming water varies periodically with time according to the expression $\gamma(t) = 2 + \sin (2t) \frac{g}{gal}$. Construct a mathematical model of this flow process and determine the amount of chemical in the pond at any time.
\end{ex}\noindent
Define $Q(t)$ as the amount of chemical in the tank at time $t$. By this definition, we see that $\frac{dQ}{dt} = $ rate coming in - rate coming out. This can be modelled as
\begin{align}
\frac{dQ}{dt} &= (5 \times 10^6 \frac{gal}{yr})\gamma(t) - (5 \times 10^6)\left(\frac{Q(t)}{10^7}\right) \\
&= (5 \times 10^6)(2 + \sin (2t)) - \frac{Q(t)}{2}
\end{align}
Let $q(t) = \frac{Q(t)}{10^6}$
$$\frac{dq}{dt} + \frac{1}{2}q = 10 + 5\sin (2t)$$
From here, simply solve the DE. That's the easy part --- constructing the DE is the hard part.
%END%
\end{document}