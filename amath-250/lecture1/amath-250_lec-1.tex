\documentclass{report}
\usepackage[margin=1in, paperwidth=8.5in, paperheight=11in]{geometry}
%Math packages%
\usepackage{amsmath}
\usepackage{amsthm}
%Spacing%
\usepackage{setspace}
\onehalfspacing
%Lecture number%
\newcommand{\lectureNum}{1}
%Variables - Date and Course%
\newcommand{\curDate}{May 2, 2017}
\newcommand{\course}{AMATH 250}
\newcommand{\instructor}{Zoran Miskovic}
%Defining the example tag%
%\theoremstyle{definition}%
\newtheorem{ex}{Example}[section]
%Setting counter given the lecture number%
\setcounter{chapter}{\lectureNum{}}
\usepackage{graphicx}
\usepackage{calc}
\usepackage{listings}

\lstset{frame=tb,
  aboveskip=3mm,
  belowskip=3mm,
  showstringspaces=false,
  columns=flexible,
  basicstyle={\small\ttfamily},
  numbers=none,
  breakatwhitespace=true,
  tabsize=3
}


\begin{document}
%Note title%
\begin{center}
\begin{Large}
\textsc{\course{} | Lecture \lectureNum{}}
\end{Large}
\end{center} 
\noindent \textit{Bartosz Antczak} \hfill
\textit{Instructor: \instructor{}} \hfill
\textit{\curDate{}}
\rule{\textwidth}{0.4pt}
% Actual Notes%
\section{Introduction}
What are differential equations? An informal definition is: \textit{a differential equation (DE) is an equation that relates an unknown function to its own derivative(s).}\\
DEs are the language of science and engineering. Fundamental laws are expressed as DEs.
\begin{ex}
Classical mechanics
\end{ex}
\begin{align}
    v(t) &= \frac{dx}{dt} \\
    a(t) &= \frac{dv}{dt} = \frac{d^2x}{dt^2}\\
    \implies F(x) &= m\frac{d^2x}{dt^2} 
\end{align}
\begin{ex}
Lotka-Volterra equation
\end{ex}\noindent
Let $x(t) = $ number of prey. Let $y(t) = $ number of predators.
\begin{align}
    \frac{dx}{dt} &= \alpha x - \beta xy \\
    \frac{dy}{dt} &= \delta xy - \gamma y
\end{align}
%END%
\end{document}
