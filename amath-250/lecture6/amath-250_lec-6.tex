\documentclass{report}
\usepackage[margin=1in, paperwidth=8.5in, paperheight=11in]{geometry}
%Math packages%
\usepackage{amsmath}
\usepackage{amsthm}
\usepackage{amssymb}
%Spacing%
\usepackage{setspace}
\onehalfspacing
%Lecture number%
\newcommand{\lectureNum}{6}
%Variables - Date and Course%
\newcommand{\curDate}{May 14, 2017}
\newcommand{\course}{AMATH 250}
\newcommand{\instructor}{Zoran Miskovic}
%Defining the example tag%
%\theoremstyle{definition}%
\newtheorem{ex}{Example}[section]
%Setting counter given the lecture number%
\setcounter{chapter}{\lectureNum{}}
\usepackage{graphicx}
\usepackage{calc}
\usepackage{listings}

\lstset{frame=tb,
  aboveskip=3mm,
  belowskip=3mm,
  showstringspaces=false,
  columns=flexible,
  basicstyle={\small\ttfamily},
  numbers=none,
  breakatwhitespace=true,
  tabsize=3
}


\begin{document}
%Note title%
\begin{center}
\begin{Large}
\textsc{\course{} | Lecture \lectureNum{}}
\end{Large}
\end{center} 
\noindent \textit{Bartosz Antczak} \hfill
\textit{Instructor: \instructor{}} \hfill
\textit{\curDate{}}
\rule{\textwidth}{0.4pt}
% Actual Notes%
\subsubsection{Last Time}
Skydiver problem and physical quantities.
\section{More on Physical Quantities}
Consistency requirements to be satisfied by all equations having physical content:
\begin{enumerate}
    \item \textbf{Principle of Dimensional Homogeneity:} one may only add, subtract, and equation quantities that have the same dimension
    \item Quantities having different dimensions may only be combined by multiplication or division
    \item The argument of a function must be dimensionless. For example, consider the value $e^{\frac{\alpha}{m}t}$. The dimension $\frac{\alpha}{m}t$ must have a dimension of [1].
    \item The value of each function must also be dimensionless
\end{enumerate}
\section{Newton's Law of Gravitation and the Problem of Escape Velocity}
Recall Newton's gravitational equation
$$F = G \frac{Mm}{r^2}$$
where $r = r(t)$ is the distance from the centre of Earth. Also, let $R$ be the radius of the Earth (constant). When $r=R$, then
$$F = -mg$$
and so
$$g = \frac{GM}{R^2}$$
In more general terms, we have
\begin{align}
    F &= G\frac{Mm}{r^2} \\
    &= m\left(\frac{MG}{R^2}\right)\frac{R^2}{r^2} \\
    &= mg\frac{R^2}{r^2}
\end{align}
From here we create our DE for $v(t)$:
\begin{align}
    m\frac{dv}{dt} &= mg\frac{R^2}{r^2} \\
    \frac{dv}{dt} &= g\frac{R^2}{r^2} \\
    \frac{d^2r}{dt^2} &= g\frac{R^2}{r^2}
\end{align}
Here we have a 2nd-order DE for $r(t)$. We can solve this by making an assumption that $v(t) = v(r(t))$ is a composite function of $t$. We utilize the chain rule:
$$\frac{dv}{dt} = \frac{dv}{dr} \cdot \frac{dr}{dt} = \frac{dv}{dr}v$$
We now sub this value into (6.5):
\begin{align}
    \frac{dv}{dr}v &= g\frac{R^2}{r^2} \\
    \int v \; dv &= -gR^2\int \frac{dr}{r^2} \\
    \frac{1}{2}v^2 &= g\frac{R^2}{r} + c
\end{align}
%END%
\end{document}