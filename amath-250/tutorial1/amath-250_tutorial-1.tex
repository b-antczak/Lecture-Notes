\documentclass{report}
\usepackage[margin=1in, paperwidth=8.5in, paperheight=11in]{geometry}
%Math packages%
\usepackage{amsmath}
\usepackage{amsthm}
\usepackage{amssymb}
%Spacing%
\usepackage{setspace}
\onehalfspacing
%Lecture number%
\newcommand{\lectureNum}{1}
%Variables - Date and Course%
\newcommand{\curDate}{May 8, 2017}
\newcommand{\course}{AMATH 250}
\newcommand{\instructor}{}
%Defining the example tag%
%\theoremstyle{definition}%
\newtheorem{ex}{Example}[section]
%Setting counter given the lecture number%
\setcounter{chapter}{\lectureNum{}}
\usepackage{graphicx}
\usepackage{calc}
\usepackage{listings}

\lstset{frame=tb,
  aboveskip=3mm,
  belowskip=3mm,
  showstringspaces=false,
  columns=flexible,
  basicstyle={\small\ttfamily},
  numbers=none,
  breakatwhitespace=true,
  tabsize=3
}


\begin{document}
%Note title%
\begin{center}
\begin{Large}
\textsc{\course{} | Tutorial \lectureNum{}}
\end{Large}
\end{center} 
\noindent \textit{Bartosz Antczak} \hfill
\textit{\curDate{}}
\rule{\textwidth}{0.4pt}
% Actual Notes%
\section{Logistic Model for Population Growth}
Define $p(t) > 0$ as the population of a particular species with respect to time $t$.\\
Define $b$ as the birth rate. We assume $b = b_0$ is constant \\
Define $d$ as the death rate. We assume $d$ is \textbf{not} a constant:
$$d(t) = d_0 + d_1p(t)$$
where $d_0, d_1 > 0$ are both constants. \\
Define our model as 
\begin{align}
\frac{dp}{dt} &= (b - d)p\\&= (b_0 - d_0)p - d_1p^2
\end{align}
Define $r = (b_0 - d_0) > 0$ as the growth rate.\\
Lastly, define $k$ as the carrying capacity of the system:
$$k = \frac{b_0 - d_0}{d_1}$$
From this, we have the following DE:
$$\frac{dp}{dt} = rp\left(1 - \frac{p}{k}\right)$$
\subsection{Qualitative analysis}
We first observe two things:
\begin{enumerate}
    \item $\frac{dp}{dt} > 0 \implies 0 < p < k$
    \item $\frac{dp}{dt} < 0 \implies p > k$
\end{enumerate}
We also see that the equilibrium solution occurs when $\frac{dp}{dt} = 0 \implies p = k.$\\Now let's solve our DE.
\subsection{Solving our DE}
It's a separable DE.
\begin{align}
    \int \frac{dp}{dt} \cdot \frac{1}{p\left(1 - \frac{p}{k}\right)} \; dt = \int r \; dt \\
    \int \frac{k}{p(k-p)} \; dp = \int r \; dt
\end{align}
\subsubsection{Aside}
We integrate the left-hand side by partial fractions:
\begin{align*}
    \frac{k}{p(k-p)} &= \frac{A}{p} + \frac{B}{k-p} = \frac{Ak - Ap + Bp}{p(k-p)} \\
    &= \frac{1}{p} + \frac{1}{k-p}
\end{align*}
Back to integrating:
\begin{align}
    \int \frac{1}{p} \; dp + \int \frac{1}{k-p} \; dp &= rt + c \\
    \ln |p| - \ln |k - p| &= rt + c \\
    \biggl\lvert\frac{p}{p-k}\biggr\rvert &= e^{rt + c} \\
    \frac{p}{p-k} &= c_2e^{rt} && \text{(Let }c_2 = \pm e^c) \\
    p &= (p-k)c_2e^{rt} \\
    p &= - \frac{kc_2e^{rt}}{1 - c^2e^{rt}} \\
    p &= \frac{kc_2e^{rt}}{c^2e^{rt} - 1}
\end{align}

%END%
\end{document}