\documentclass{report}
\usepackage[margin=1in, paperwidth=8.5in, paperheight=11in]{geometry}
%Math packages%
\usepackage{amsmath}
\usepackage{amsthm}
\usepackage{amssymb}
%Spacing%
\usepackage{setspace}
\onehalfspacing
%Lecture number%
\newcommand{\lectureNum}{21}
%Variables - Date and Course%
\newcommand{\curDate}{June 18, 2017}
\newcommand{\course}{AMATH 250}
\newcommand{\instructor}{Zoran Miskovic}
%Defining the example tag%
%\theoremstyle{definition}%
\newtheorem{ex}{Example}[section]
%Setting counter given the lecture number%
\setcounter{chapter}{\lectureNum{}}
\usepackage{graphicx}
\usepackage{calc}
\usepackage{listings}

\lstset{frame=tb,
  aboveskip=3mm,
  belowskip=3mm,
  showstringspaces=false,
  columns=flexible,
  basicstyle={\small\ttfamily},
  numbers=none,
  breakatwhitespace=true,
  tabsize=3
}


\begin{document}
%Note title%
\begin{center}
\begin{Large}
\textsc{\course{} | Lecture \lectureNum{}}
\end{Large}
\end{center} 
\noindent \textit{Bartosz Antczak} \hfill
\textit{Instructor: \instructor{}} \hfill
\textit{\curDate{}}
\rule{\textwidth}{0.4pt}
% Actual Notes%
\section{Laplace Transformations (Ch. 4)}
Recall our \textit{swing problem} that we covered in rotational dynamics (for which I did not write notes):
$$y^{\prime\prime} + 2\gamma y^\prime + \omega_0^2y = F(t)$$
We studied when $F(t) = f_0 \cos (\omega t)$. 
\subsubsection{Notation}
We introduce $\mathcal{L}$ as an operator which takes a function $y(t)$ and produces another function $Y(s)$. This function simplifies hard DE problems.
\subsubsection{Formal Definition}
Given a real-(or complex-)valued function $y(t)$, defined on $t \in [0, \infty)$, the Laplace Transform $\mathcal{L}[y(t)]$ is defined as
$$Y(s) = \int_0^\infty e^{-st}y(t) \; dt$$
for all values of $s$ for which the improper integral exists. \\
Even though $y(t)$ and $s$ may be generally complex-valued, we'll assume for them to be \textbf{real} values, so $Y(s)$ is also real valued.
\begin{ex}
Find $\mathcal{L}[y(t)]$ for $y(t) = e^{\alpha t}$ with $\alpha$ being constant
\end{ex}\noindent
\begin{align*}
Y(s) &= \lim_{T \to \infty} \int_0^T e^{-st}e^{\alpha t} dt \\
&= \lim_{T \to \infty} \int_0^T e^{(\alpha - s)t}dt \\
&= \lim_{T \to \infty} \begin{cases}
\frac{e^{(\alpha- s)t}}{\alpha - s}\biggl|_0^T & s \neq \alpha \\
T & s = \alpha \text{ (not defined})
\end{cases}
\end{align*}
For $s \neq \alpha$:
\begin{align*}
\lim_{T \to \infty} \frac{1 - e^{-(s - \alpha)T}}{s - \alpha} = \begin{cases}\frac{1}{s - \alpha} & s > \alpha \\ DNE & s  < \alpha\end{cases}
\end{align*}
And so 
$$\mathcal{L}[e^{\alpha t}] = Y(s) = \frac{1}{s - \alpha} \qquad \qed$$
\subsubsection{Linearity}
If $\mathcal{L}[y_1(t)]$ and $\mathcal{L}[y_2(t)]$ exist, then
$$\mathcal{L}[c_1y_1(t) + c_2y_2(t)] = c_1 \mathcal{L}[y_1(t)] + c_2 \mathcal{L}[y_2(t)]$$
\subsubsection{Conditions to Exist}
$\mathcal{L}[f]$ exists if
\begin{itemize}
\item $f(t) \in O(e^{\alpha t})$ for some $\alpha \in \mathbb{R}$
\item $f$ is a piecewise continuous function
\end{itemize}
%END%
\end{document}